\documentclass[a4paper, 11pt]{article}
\usepackage{geometry}
\usepackage{fancyhdr}
\usepackage{fancybox} 
\usepackage{hyperref}
\usepackage{amsmath}
\usepackage{amsfonts}
\usepackage{amsthm}
\usepackage{lipsum} 
\usepackage{natbib}
\usepackage{libertine}
\usepackage{sectsty}
\allsectionsfont{\sffamily}

\usepackage{hyperref}
\hypersetup{
    colorlinks=true,
    linkcolor=RoyalBlue,
    citecolor=green,    
    filecolor=magenta,  
    urlcolor=Mahogany   
}
\geometry{a4paper, margin=25mm}

\usepackage[most]{tcolorbox}
\usepackage[svgnames, dvipsnames]{xcolor}
\newtheorem{theorem}{Theorem}[section]
\newtheorem{lemma}[theorem]{Lemma}
\newtheorem{corollary}[theorem]{Corollary}
\newtheorem{example}[theorem]{Example}
\newtheorem{question}[theorem]{Question}

% Base style
\tcbset{
  base theorem box/.style={
    enhanced,
    colframe=blue!50!black,     
    colback=blue!5!white,       
    coltitle=white,             
    fonttitle=\bfseries,        
    title filled,               
    arc=2mm,                    
    boxrule=0.8pt,              
    drop shadow=black!75!white, 
    boxsep=2mm,                 
    left=2mm,right=2mm,
    top=2mm,bottom=2mm,
    before skip=12pt,
    after skip=12pt,
    attach boxed title to top left={yshift=-2mm, xshift=2mm},
  }
}


% Specific styles with distinct colors
\newtcbtheorem[number within=section]{Theorem}{Theorem}
{base theorem box, colback=blue!5, colframe=blue!75!black, colbacktitle=blue!5, coltitle=blue!75!black}{thm}

\newtcbtheorem[use counter from=Theorem]{Lemma}{Lemma}
{base theorem box, colback=green!5, colframe=green!50!black, colbacktitle=green!5, coltitle=green!50!black}{lem}

\newtcbtheorem[use counter from=Theorem]{Corollary}{Corollary}
{base theorem box, colback=purple!5, colframe=purple!60!black, colbacktitle=purple!5, coltitle=purple!60!black}{cor}

\newtcbtheorem[use counter from=Theorem]{Example}{Example}
{base theorem box, colback=orange!5, colframe=orange!70!black, colbacktitle=orange!5, coltitle=orange!70!black}{ex}

\newtcbtheorem[use counter from=Theorem]{Question}{Question}
{base theorem box, colback=pink!5, colframe=pink!70!black, colbacktitle=pink!5, coltitle=pink!70!black}{qst}

\pagestyle{fancy}
\fancyhf{}
\rhead{\thepage}
\chead{\title{}}
\lhead{\textsc{Course-Code Course Name}}

\begin{document}

\begin{Sbox}
    \begin{minipage}{\textwidth}
        \thispagestyle{empty}
        \begin{center}
            \textsc{\Large Course-Code: Course Name} \\[0.5cm] 
            \textbf{\Large \sffamily Assignment Title}    
        \end{center}
        \vfill
        \begin{tabular}{p{0.3\textwidth} p{0.6\textwidth}}
            \textbf{Author:} & Rishabh Sharad Pomaje \\ 
            \textbf{Instructor:} & Instructor Name \\ 
            \textbf{Roll Number:} & Your Roll Number \\ 
            \textbf{Email:} & \href{mailto:rishabhp@stanford.edu}{\texttt{rishabhp@stanford.edu}}
        \end{tabular}
    \end{minipage}
\end{Sbox}

\begin{center}
    \fbox{\TheSbox}
\end{center}
\hrule

\section{Introduction}
\lipsum[2]

\begin{Theorem}{Rishabh}{ris}
Let \( f \) be continuous. Then \( f \) is integrable.
\end{Theorem}

\begin{Lemma}
This result helps prove the main theorem.
\end{Lemma}

\begin{Corollary}
A differentiable function is continuous.
\end{Corollary}

\begin{Example}
Let \( f(x) = x^2 \), which is differentiable on \( \mathbb{R} \).
\end{Example}

\begin{Question}
What happens if the function is not continuous?
\end{Question}


\section{Main Content}
\subsection{Subsection 1}
\lipsum[3]

\subsection{Subsection 2}
\lipsum[4]

\section{Conclusion}
\lipsum[5]

% References 
\bibliographystyle{plainnat}
\bibliography{references}

\end{document}
